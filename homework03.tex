\documentclass{ctexart}

\usepackage{graphicx}
\usepackage{amsmath}
\usepackage{xltxtra}

\title{作业三:介绍Linux工作环境}
\author{蔡聪聪 \\ 信息与计算科学 3180102279}

\begin{document}

\maketitle

\section{Linux的系统信息}
\raggedright
\textbf{计算机类型:}\verb|x86_64|\quad \textbf{计算机名:}\verb|ubuntu|\quad \textbf{操作系统名称:}\verb|Linux|\\
\textbf{操作系统的发行编号:}\verb|4.15.0-188-generic|\\
\textbf{系统版本与时间:}\verb|199-Ubuntu SMP Wed Jun 15 20:42:56 UTC 2022|

\section{对系统做的调整}

\textbf{对系统的主要调整:}\\
将系统语言设置成中文,未做其余调整\\
\textbf{安装的软件:} \\
\verb|emacs fcitx g++ latex python3 git okular synaptic|\\
\textbf{额外的配置工作:}\\
\verb|fcitx|下安装了 \verb|google|拼音\\
调整了 \verb|emacs|的配置\\
使用 \verb|git|将本地仓库与 \verb|github|关联


\section{下一步的工作}

\subsection{未来半年内将在什么场合下使用Linux环境}
预计未来将在两个场合使用\verb|Linux|环境:
\begin{itemize}
\item 用 \LaTeX 来组织报告和论文
\item 用 \verb|C++|编写程序解数学问题
\end{itemize}


\subsection{目前的工作环境与未来需求的契合度}
目前工作环境不符合未来需求.\\
目前各类软件配置还是缺省配置,能正常运行.但由于自身熟练度较低,工作效率低下.\\
计划在更熟悉\verb|Linux|工作环境后自行调整系统及各软件的配置

\section{如何保证工作系统中的代码, 文献和工作结果的稳定和安全}



\end{document}
